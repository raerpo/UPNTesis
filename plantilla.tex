% Plantilla para la realización de trabajos de grado de la Universidad Pedagógica Nacional
% Basada en la clase y plantilla creada por Juan Alberto Ramírez Macías - juan.ramirez@upb.edu.co
% V 1.0 / Rafael Ernesto Poveda - raerpo@gmail.com
% 06/05/12
% Codigo disponible en https://github.com/raerpo/UPNTesis

\documentclass[]{UPN}
\usepackage{graphicx}
\usepackage{amsfonts}
\usepackage{amsmath}
%\usepackage{dropping}
\usepackage[usenames,dvipsnames]{color}

%Paquete de configuración de los hipervinculos internos del pdf
\usepackage{hyperref}
\hypersetup{colorlinks=true,linkcolor=black,linktoc=all,citecolor=black,}

%<---------------------------- Desde aqui se modifica la plantilla ---------------------------------------->
%<--------------------------------------------------------------------------------------------------------->

\title{PLANTILLA \LaTeX ~PARA LA CREACIÓN DE TRABAJOS DE GRADO DE LA LICENCIATURA EN ELECTRÓNICA UNIVERSIDAD PEDAGÓGICA NACIONAL}
\author{PRIMER AUTOR DEL TRABAJO\\SEGUNDO AUTOR DEL TRABAJO\\TERCER AUTOR DEL TRABAJO} 
\degree{Trabajo de grado para optar al título de Licenciado en electrónica}
\director{NOMBRE DEL ASESOR\\Ingeniero en \LaTeX}

\begin{document}

\portada
\aprobacion

\begin{agradecimientos}
\textbf{(Esta sección es Opcional)}~En ella el autor o autores del trabajo, dedican su  trabajo en forma especial a personas o entidades. Se 
conservan las márgenes de las demás páginas preliminares. 
En ella el autor (es) agradece(n) a las personas o instituciones, instrumentos empleados, entre otros, que 
colaboraron en la realización del trabajo. Deben aparecer los nombres completos, los cargos y aporte al 
trabajo de las personas mencionadas.
\end{agradecimientos}

\begin{RAE}
El resumen analítico en educación debe aparecer antes de la tabla de contenido (según requerimiento de la 
Biblioteca Central) Su extensión es de 500 palabras máximo para este tipo de documentos y debe contener: 
tipo de documento (trabajo de grado), acceso al documento (Universidad Pedagógica Nacional), título del 
documento, autor, publicación (fecha), unidad patrocinante (Universidad Pedagógica Nacional), palabras
claves, descripción, fuentes (autores principales presentados en la bibliografía), contenidos (apartados 
relevantes redactados en prosa, no enunciados), metodología, conclusiones y fecha de elaboración del 
resumen.
\end{RAE}

\tabladecontenido
\listadefiguras
\listadetablas

\section{INTRODUCCIÓN}
Esta plantilla fue creada con el fin de facilitar la creación de los trabajos de grado de la Universidad Pedagógica Nacional, especialmente, los de la licenciatura en electrónica. La clase \LaTeX~\textbf{"UPN.cls"} esta basada en la clase \textbf{"upnthesis.cls"}\footnote{http://goo.gl/0vblP} creada por \textbf{Juan Alberto Ramírez Macías} de la Universidad Pontificia Bolivariana. Esta clase usa por defecto tamaño de letra {\tt 12pt}, interlineado de {\tt 1.5}, y hojas tamaño carta\footnote{Estos valores se pueden modificar desde el código fuente de la clase UPN.cls}.

Para definir secciones y sub-secciones se usan los comandos "section", "subsection", "subsubsection", y  "paragraph". De esta manera se puede conseguir un nivel de identación de hasta 4, tal y como se ve en la sección \ref{Ejemplo_seccion}

La inclusión de imágenes y tablas se realiza de la misma manera como se haría en cualquier clase \LaTeX.

\section{Ejemplo de seccion} \label{Ejemplo_seccion}
Este es el texto que va adentro de un {\tt section}.
\subsection{Ejemplo de un subseccion}
Este es el texto que va adentro de un {\tt subsection}.
\subsubsection{Ejemplo de subsubseccion}
Este es el texto que va adentro de un {\tt subsubsection}.
\paragraph{Ejemplo de paragraph}
Este es el texto que va adentro de un {\tt paragraph}.

\end{document}
